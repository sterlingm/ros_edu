\documentclass[12pt]{article}
\usepackage[margin=1in]{geometry} 
\usepackage{amsmath,amsthm,amssymb,amsfonts}
\usepackage{enumerate} 
\usepackage{hyperref}
\hypersetup{
colorlinks=true,
urlcolor=blue
}
 
\newenvironment{task}[2][Task]{\begin{trivlist}
\item[\hskip \labelsep {\bfseries #1}\hskip \labelsep {\bfseries #2.}]}{\end{trivlist}}

 
\begin{document}
 
\title{Turtlebot Tele-operation}
\author{Author: Sterling McLeod}
\maketitle

\subsection*{Introduction}

The assignment will train you to move the Turtlebot 2 platform via velocity commands. First, you will start processes on the Turtlebot by using a \emph{launch} file. After the Turtlebot has been started, you will publish a specific msg, \emph{geometry\_msgs/Twist}, to drive the robot. Lastly, you will design a basic keyboard interface for a user to operate the Turtlebot through a keyboard. By doing all of this, you will become more familiar with publishing and subscribing messages and how to structure a ROS node. To complete this assignment, you should already be familiar with creating a ROS package, building a ROS node, and how to publish and subscribe to messages.



\subsection*{System Info}
\emph{Operating System, ROS Version}: Ubuntu 14.04, Indigo

\subsection*{Supplemental Material}

Visit \href{http://wiki.ros.org/ROS/Tutorials}{http://wiki.ros.org/ROS/Tutorials} for a list of tutorials on ROS. For this assignment, the following tutorials will be most relevant:

\begin{enumerate}
\item Understanding ROS Nodes
\item Understanding ROS Topics
\item Creating a ROS msg and srv
\item Writing a Simple Publisher and Subscriber (C++), Writing a Simple Publisher and Subscriber (Python)
\item Using rqt\_console and roslaunch
\end{enumerate}

\subsection*{Tasks}

\begin{task}{1. Run and observe the keyboard tele-operation provided by ROS.}
\end{task}
\begin{enumerate}
\item Change permissions on the interface between the robot and computer: "sudo chmod 777 /dev/ttyUSB0"
\item Run the "minimal.launch" file in the package turtlebot\_bringup
\item Drive the robot around with the keyboard interface.
\end{enumerate}


\begin{task}{2. Create a ROS package implementing keyboard tele-operation}
\end{task}

\begin{enumerate}
\item Accept keyboard commands for the following:
	\begin{enumerate}
	\item Move forward or backwards for X number of seconds at a fixed speed
	
		\begin{enumerate}
		\item Limit the time to 2 seconds
		\end{enumerate}
	
	\item Move forward or backwards X distance
	
		\begin{enumerate}
		\item Limit the distance to 0.33m
		\end{enumerate}
		
	\item Turn 90 degrees clockwise and counter-clockwise
	
	\item Turn X degrees
	
		\begin{enumerate}
		\item Limit to 90 degrees
		\end{enumerate}
		
	\item Change the linear and angular velocities of the robot
		\begin{enumerate}
		\item Limit the linear velocity to $0.33\frac{m}{s}$ and angular velocity to $0.52\frac{rad}{s}$.
		\end{enumerate}
		
	\item The keys used for tele-operation do not need to match the keys used in the keyboard\_teleop.launch file.
	\item The robot does not need to stop smoothly.
	\end{enumerate}
	
	\item Create a launch file to run all nodes necessary for your keyboard tele-operation node.
	
	\item Run \emph{rqt\_graph} with your launch file running and save an image of it (there's an icon in the top-right of the window for this).
	
	\item Create a README file explaining how to run your code.
	
\end{enumerate}
 
\end{document}